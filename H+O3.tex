\documentclass[iop]{emulateapj}
\usepackage{natbib}
\usepackage{amsmath}
\usepackage{graphicx}
\usepackage{hyperref}
%\usepackage{breakurl}
\usepackage{verbatim}
\usepackage{epsfig}

\shorttitle{H+O$_3$}
\shortauthors{He \& Vidali}

\begin{document}

\title{Formation of water on warm silicates}
    
    \author{Jiao He and Gianfranco Vidali}
\affil{Physics Department, Syracuse University,
    Syracuse, NY 13244, USA}
    \email{gvidali@syr.edu}

\begin{abstract}
When dust grains are at a temperature higher than the one they have in dense clouds, and when H, H$_2$, and O$_2$ have a negligible residence time on grains, formation of water should still  be possible  through the hydrogenation of ozone, i.e., via the H+O$_3$ channel. We measured the cross-section for the reaction of H and  D with an O$_3$  layer on an amorphous silicate surface at 50 K and the subsequent formation of water. We find that these cross-sections are indicative of the direct Eley-Rideal reaction mechanism instead of the hot-atom one. We determined that the OH desorption energy from an amorphous silicate surface is  at least 145 meV (1680 K), and possibly as high as 410 meV (4760 K). This extends the temperature range for the efficient formation of water on grains from about 30 K to at least 50 K, and possibly as high as 160 K. We don't find evidence that water molecules leave the surface upon formation. Instead, through a thermal programmed desorption experiment, we find that water formed on the surface of an amorphous silicate desorbs at around 160 K. Information obtained through these experiments should help theorists evaluate the relative contribution of water formation on warm grains vs. in the gas-phase.
\end{abstract}

\keywords{ISM: molecules --- ISM: atoms}

\section{Introduction}

Atomic hydrogen related chemistry on interstellar dust grain surfaces plays an important role in the chemical evolution of the interstellar medium. Being the most abundant atomic species in the universe, atomic hydrogen participates in reactions to make many molecules that enter in key physical and chemical processes \citep{Vidali2013}.

Almost all the grain surface reactions in the interstellar medium (ISM) depend on hydrogen in a direct or indirect way. Water formation is one such important reaction. Water acts as coolant and a tracer in gravitationally collapsing clouds \citep{Neufeld1995,Mottram2013} and provides, as the main constituent of ices coating dust grains, the medium for the chemical enrichment of many space environment, from dense clouds to protoplanetary disks \citep{Kristensen2011,Hogerheijde2011}. It is well-known that gas-phase reactions alone cannot account for the abundance of water in the ISM, and reactions taking place on dust grain surfaces must be considered \citep{Hasegawa1992,Roberts2002}.  Following a reaction network put forward by  \citet{Tielens1982}, there are three main routes to form water via hydrogenation of oxygen (O+H$\rightarrow$ OH, OH+H $\rightarrow$ H$_2$O, of molecular oxygen (O$_2$+H $\rightarrow$ HO$_2$+H $\rightarrow$ HO$_2$ + H $\rightarrow$ H$_2$O$_2$ + H $\rightarrow$ H$_2$O + OH) and of ozone (O$_3$+H $\rightarrow $ OH +O$_2$ followed by reactions with H or H$_2$). In the past decade, there has been an intense interest in studying in the laboratory which of these reaction paths are most likely to occur in a given space environment. Most of the work was done on model ices (water, oxygen, ozone) at low temperature (less than 30 K). Formation of water using atomic hydrogen and oxygen was studied by \citet{Dulieu2010} and \citet{Jing2011} on a water ice surface and on a surface of an amorphous silicate, respectively. Formation of water via hydrogenation of a molecular oxygen ice was investigated by \citet{Ioppolo2008} and \citet{Miyauchi2008}, while \citet{Romanzin2011} and  \citet{Mokrane2009} used ozone made in-situ or deposited ozone on an ice to form water by the reaction H+O$_3$. \citet{Oba2011} studied the formation of water on a 40-60K substrate through the OH+OH $\rightarrow$ H$_2$O$_2$ channel.
 
However, water formation proceeds also in warmer regions; some of it occurs via gas-phase  endothermic or ion-atom/molecule reactions\citep{Glassgold2009,Hollenbach2012}. In warm regions the O$_2$ hydrogenation channel in surface reactions is not likely to be important since O$_2$ leaves the surface of ice at around 30 K. The other two channels (hydrogenation of O and O$_3$) involve the radical OH:  OH+H$\rightarrow$H$_2$ or OH+H$_2\rightarrow$H$_2$O+H, for O and O$_3$ hydrogenation, respectively. Therefore it is critical to understand what  the residence time of OH on grain surfaces is and the mechanisms of reactions on warm grains where the residence time of H or H$_2$ can be very short. 

In diffuse clouds, OH is formed via a series of reactions starting with the O$^+$ ion (for example:  H$^+$+O $\rightarrow$ O$^+$+H, O$^+$ + H$_2 \rightarrow$ OH$^+$ + H). In the gas-grain chemical modeling of \citet{Hasegawa1992} for dense clouds, OH desorption energy was chosen to be 1259 K, which is adapted from \citet{Allen1977}. A simple calculation based on this value predicts that the residence time of OH on grain surface is less than 0.1 second at 50 K if we assume a pre-exponent $\nu=10^{12} s^{-1}$ in the standard expression $t=\nu^{-1}\exp(-E_{des}/k_BT)$, where $E_{des}$ is the energy necessary for the molecule to leave the surface (desorption energy) and $T$ is the temperature of the surface. Therefore, assuming these values, water formation on dust at above 50 K is not efficient enough if we assume that reactions occur on surfaces with atoms or radicals thermally accommodated to them, which disagrees with astronomical observations. 

There are three known mechanisms for atomic hydrogen to participate in reactions on grain surfaces: Langmuir-Hinshelwood (L-H), Eley-Rideal (E-R) , and hot-atom  (sometimes in the literature the hot-atom mechanism is also called indirect Eley-Rideal mechanism) \citep{Kolasinski2008}. In the L-H mechanism gas phase atoms stick to the surface and accommodate thermally to the surface; then they diffuse and react with another atom or molecule. In astrochemistry, the L-H mechanism is invoked to explain most of the experimental data on the formation of molecular hydrogen, CO$_2$, formaldehyde and water on surfaces of ices and silicates at low temperature \citep{Vidali2013b}. However, on pre H-dosed graphitic and polycyclic aromatic hydrocarbon (PAH) surfaces\citep{Hornekaer2006,Baouche2006,Baouche2009,Mennella2008,Mennella2012}, on tholins (analogs of aerosol particles in Titan's atmosphere) \citep{Sekine2008},  the formation of H$_2$ or of its isotopes was shown to proceed via the Eley-Rideal or hot-atom mechanisms. \citet{lemaire2010} showed that on a surface of a silicate film, formation of D$_2$ occurs up to a surface temperature of 70K, presumably via the hot-atom mechanism. In this case the molecule so formed leaves the surface of the silicate promptly and in a ro-vibrational excited state. 
In the Eley-Rideal case \citep{Harris1981}, the impinging particle reacts with another one chemisorbed on the surface without prior accommodation. Whether and how the molecule leaves the surface in the Eley-Rideal event has not firmly established from a theoretical standpoint and  might depend on the conditions of the surface, the binding energy and other factors. Further details can be found in a  recent review \citep{Vidali2013c} that summarizes both experimental and theoretical studies of H interaction with and H$_2$ formation on surfaces of materials of interest to astrochemsitry. 

In studies of well-characterized single-crystal metal surfaces, it has been shown that in the Eley-Rideal case the molecule retains much of the energy gained in the reaction and  leaves the surface in a ro-vibrational excited state \citep{Rettner1992} and in a angularly focused direction \citep{Quattrucci2005}. In the hot-atom case the incoming particle lands on a  bare spot of the surface and travels on it at super thermal speed. Some of this energy comes from the binding energy with the surface. If it encounters another reactant, it might form a bond. As in the Eley-Rideal case, there isn't a complete accommodation  of the particle with the surface. The two mechanisms are distinguished in practice by the different cross-sections for reaction, but the distinction is often not so clear-cut \citep{Kim1999}. In the Eley-Rideal case, it is of the order of the size of the geometric area of the atom or radical already present on the surface, i.e. \AA$^2$. In the second case, it is larger, of the order of a few times the cross-section for the Eley-Rideal case \citep{Guvenc2001}. In dense clouds where the atomic hydrogen residence time is sufficiently long, the L-H mechanism dominates while in regions where the atomic hydrogen residence time is too short, only the E-R and hot-atom mechanisms are possible. Whether atomic hydrogen takes part in grain surface reactions via the E-R or the hot-atom mechanism when the dust grain temperature is relatively high, e.g., higher than 50 K, it is still an open question. 

In this paper we report on the formation of water on warm grains. We investigated  the formation of water via the OH reaction with H/H$_2$. OH was obtained by the reaction of O$_3$ with H. The mechanism of  reaction of atomic hydrogen with ozone was quantified.  This was done  in a single set of experiments on the  surface of an amorphous silicate film at 50 K via  O$_3$+H sequential exposure experiments. The following reactions are of interest:
\begin{equation}
 \text{O}_3+\text{H/D}\rightarrow \text{OH/OD}+\text{O}_2
 \label{eq:o3+h}
\end{equation}
\begin{equation}
 \text{OH/OD}+\text{H/D}\rightarrow\text{H}_2\text{O}/\text{D}_2\text{O}
 \label{eq:oh+h}
\end{equation}
\begin{equation}
  \text{OH/OD}+\text{H}_2\text{/D}_2\rightarrow\text{H}_2\text{O}/\text{D}_2\text{O}+\text{H/D}
   \label{eq:oh+h2}
\end{equation}
By measuring the destruction rate of ozone through reaction \ref{eq:o3+h}, we  get  the reaction cross-section and thus infer whether atomic hydrogen takes part in surface reaction via the E-R or hot-atom mechanisms. In studying reactions \ref{eq:oh+h} and \ref{eq:oh+h2}, we  obtain information regarding the water formation rate and  the residence time of OH on grain surfaces. 

In the next section the experimental apparatus and measuring methods are presented, followed by the experimental results. In Section \ref{sec:analysis} we present an analysis of the experimental data in order to obtain  estimates of the cross-section of the reaction with atomic hydrogen and of the OH residence time. The water formation rate is also discussed. In Section \ref{sec:sum} we explore how our findings can be used to learn about water formation in actual ISM environments . 


\section{Experimental}
\label{sec:exp}
The apparatus was described elsewhere \citep{He2011,Jing2013}; here we summarize the main features that are important for this study. The experiments were carried out in an ultra-high vacuum main chamber connected to an atomic/molecular beam line. The main chamber is pumped by a combination of a cryopump, turbo pump, and ion pump and can reach $2\times10^{-8}$ pascals after a bake-out. At the center of the chamber there is a 1 $\mu m$ thick amorphous silicate thin film sample grown on a 1 $cm^2$ gold plated copper disk by electron beam physical vapor deposition technique. The detailed preparation and characterization of the sample can be found in \citet{Jing2013}.  The sample can be cooled to 8 K by a liquid helium-cooled sample holder and be heated up to 400 K by a cartridge heater behind the sample. A triple-pass Hiden quadrupole mass spectrometer (QMS) is mounted on a rotary platform to record desorbed species from the sample surface or to measure the beam composition. The beam line has three differentially pumped stages. A radio-frequency dissociation source is mounted in the first stage of the beam line. By direct beam flux measurement and calibration experiments we found that the dissociation rate for deuterium and oxygen are about 45\% and 25\% respectively, corresponding to a flux intensity of $7.6\times 10^{11}\ cm^{-2}s^{-1}$ and $5.0\times 10^{11}\ cm^{-2}s^{-1}$ for O$_2$ and O (In the following, ``O" refers to $^{18}$O unless specified otherwise), $1.5\times10^{13}\ cm^{-2}s^{-1}$ and $9.3\times10^{12}\ cm^{-2}s^{-1}$ for D$_2$ and D respectively. The uncertainty due to the variation of the beam flux is less than 10\%. In this study we used $^{18}$O$_2$ instead of $^{16}$O$_2$. Because water is still present as a background gas even in a well-baked chamber, it is important to use isotopically labeled gases. This gives us the ability to measure very small amount of water produced on the sample. Because the measurement of  H (amu 1) and H$_2$ (amu 2) from the beam is not accurate due to the presence of background signals at these masses in the chamber. We use the same source settings (power in the dissociating RF source and pressure readings of a gauge sensitive to the thermal conductivity of gases) as in the experiments with D/D$_2$, and assume that the dissociation rate of H$_2$ is the same as that of D$_2$. Based on a correction of the different thermal conductivity of H$_2$ and D$_2$, the H$_2$ and H flux intensities can be calculated to be $1.1\times10^{13}\ cm^{-2}s^{-1}$ and $6.6\times10^{12}\ cm^{-2}s^{-1}$, respectively. When dissociating D$_2$, amu 20 is detected in the direct beam with an intensity 1.6\% of the D flux.  This is due to contamination in the foreline manifold or beam line. The D$_2\,^{16}$O that forms is subtracted from the observed signals of amu 20. 

In thermal programmed desorption (TPD) experiments the QMS measurement yields of different desorbed species cannot be compared directly unless corrected for QMS ionization efficiency. The ionization efficiency of gas species is inversely proportional to the speed of desorbed molecules $v\propto \sqrt{T/m}$, where $T$ and $m$ are the temperature and mass of the desorbing molecule respectively. $T$ is assumed to be the same as the surface temperature at which desorption peaked. The ionization efficiency also depend on the molecular species. Thus we do separate calibration experiments to find out the relative ionization of different molecular species. Without cooling down the sample, the chamber is filled with room temperature gas D$_2$, O$_2$, or D$_2\,^{16}$O to certain pressures and the QMS counts are recorded as a function of pressure. The pressure is measured by a ionization pressure gauge; calibration factors for different gas species are considered. The relative ionization efficiency is calculated to be D$_2$:O$_2$:D$_2\,^{16}$O=5.57:5.43:4.00. A QMS spectrum is taken when the sample is at room temperature to check the background contamination for different species and to check the H$_2$O breaking up percentage in the QMS ionizer. The highest peak is H$_2\,^{16}$O, which breaks up into $^{16}$OH and H in the ionizer, the ratio between amu 17 and amu 18 is about 0.25. We assume this ratio also applies to the breaking up of D$_2\,^{18}$O and H$_2\,^{18}$O. 

Ozone is formed via the reaction O+O$_2\rightarrow$O$_3$ on the surface. The procedure of ozone preparation was discussed in detail in \citet{He2014}. To summarize: the amorphous silicate sample kept at 30 K is exposed to a dissociated oxygen beam for 20 minutes, then warmed up to 50 K and kept at 50 K for 2 minutes to desorb the remaining O$_2$. From the calibration procedures, the amount of ozone formed is estimated to be 0.95$\pm$0.1 monolayer. Analysis in \citet{He2014} shows that the ozone sample prepared in this method is not mixed with O$_2$ or O. Ozone (amu 54) can break upon electron impact into O (amu 18) and O$_2$ (amu 36). The amu(36)/amu(54) ratio depends on the design and setting of the ionizer in the QMS. For the measurements in this study, the ratio is about 16, so the amu 36 amount can well represent the amount of ozone yield. With the sample covered with ozone, we keep the sample at 50 K and expose it to the dissociated hydrogen/deuterium beam for a certain time. To obtain a reproducible heating slope rate at ozone desorption temperature (start at about 60 K), the sample is cooled down to below 30 K and then heated up again to thermally desorb the reaction products and the residue ozone (measured as amu 36 in QMS). A typical sequence of experimental procedures is illustrated in Figure \ref{fig:exp_procedure}. 

\begin{figure}
\epsscale{1.2}
\plotone{./figures/exp_procedure.eps}
\caption{Illustration of a typical experimental procedure. The temperature curve is taken from a typical experiment of ozone hydrogenation. In the first stage (T = 30 K) the sample surface is exposed to $^{18}$O/$^{18}$O$_2$ beam for 20 minutes. In the second stage the sample is kept at 50 K for 2 minutes without any beam exposure to desorb the residual $^{18}$O$_2$. In the third stage the sample is cooled down to below 30 K in order to obtain reproducible heating ramp rate. The last stage is the TPD stage in which the sample is heated up to about 190 K. }
\label{fig:exp_procedure}
\end{figure}


\section{Results}
\label{sec:results}
In the O$_3$+H sequential deposition experiments, the following masses are recorded by the QMS in the TPD stage: amu 19 which is  OH formed via O$_3$+H$\rightarrow$OH+O$_2$ or H$_2$O fragmentation in the QMS ionizer; amu 20 which is  H$_2$O formed via OH+H$\rightarrow$H$_2$O or OH+H$_2\rightarrow$H$_2$O+H, and amu 36 that comes from the fragmentation in the QMS ionizer of the remaining ozone. In the O$_3$+D sequential deposition experiments, the following masses are recorded: amu 20 which is OD formed via O$_3$+D$\rightarrow$OD+O$_2$ or D$_2$O/HDO fragmentation in the QMS ionizer; amu 21 which is HDO coming from the H/D exchange reaction H$_2\,^{16}$O+D$_2\,^{18}$O$\rightarrow$HD\,$^{16}$O+HD\,$^{18}$O or OD+H$_2$(from vacuum background)$\rightarrow$HDO+H; amu 22 which is D$_2$O formed via OD+D$\rightarrow$D$_2$O or OD+D$_2\rightarrow$D$_2$O+D; and amu 36 from  ozone. The amu 36 ozone peak is centered at around 80 K while all the other masses peak at about 160 K. H$_2$O$_2$ and D$_2$O$_2$ (amu 38 and 40) are checked occasionally, but are not discernible from background. Figure \ref{fig:plot_hiden} shows a typical TPD spectra after the O$_3$+D sequential experiment. Amu 20 and 36 have a higher background than amu 21 and 22. The increase of amu 36 after 170 K is due to the O$_2$ desorption from the sample holder. 

We repeat the ozone formation procedure (about 0.95$\pm$0.1 ML of ozone is formed for each run of experiment) but change the H/D exposure length. Figure \ref{fig:H_O3} and Figure \ref{fig:D_O3} show TPD yields for different species after sequential deposition of O$_3$+H and O$_3$+D respectively. In Figure \ref{fig:D_O3} the amu 20 contribution from the beam line (D$_2\,^{16}$O) has already been subtracted (see Section \ref{sec:exp}). 

\begin{figure}
\epsscale{1.1}
\plotone{./figures/plot_hiden.eps}
\caption{A typical TPD spectra after $^{18}$O$_3$+D sequential deposition as illustrated in Figure \ref{fig:exp_procedure}. Amu 20, 21, 22, and 36 are OD,  HDO, D$_2$O, and O$_2$ (ozone fragmentation) respectively.}
\label{fig:plot_hiden}
\end{figure}

\begin{figure}
\epsscale{1.1}
\plotone{./figures/H+O3.eps}
\caption{TPD yields of masses amu 19, 20, and 36 for $^{18}$O$_3$+H sequential deposition experiments as illustrated in Figure \ref{fig:exp_procedure}. The H exposure lengths are 0 min, 4 min, 8 min, 12 min, 16 min, 24 min, 32 min, and 48 min, respectively.}
\label{fig:H_O3}
\end{figure}


\begin{figure}
\epsscale{1.1}
\plotone{./figures/D+O3.eps}
\caption{TPD yields of masses amu 20, 21, 22 and 36 for sequential deposition experiments $^{18}$O$_3$+H as illustrated in Figure \ref{fig:exp_procedure}. The D exposure lengths are 0 min, 5 min, 10 min, 15 min, 20 min, 30 min, 45 min, 60 min, 75 min, and 90 min, respectively. The amu 20 contribution from the beam line (D$_2\,^{16}$O) has already been subtracted from the yields.}
\label{fig:D_O3}
\end{figure}





\section{Analysis and discussion}
\label{sec:analysis}
\subsection{H/D gas-grain reaction mechanism}
When the ozone-covered sample is exposed to the H/D beam, the reaction O$_3$+H/D$\rightarrow$OH/OD+O$_2$ which has no reaction energy barrier \citep{Mokrane2009} takes place. The reaction product O$_2$ desorbs upon reaction because of the very short residence time \citep{Jing2012}. The QMS data recorded in the H/D exposure stage shows that amu 36 signal increases above background, which confirms the desorption of O$_2$ upon formation. The ozone destruction rate is proportional to the ozone coverage, H/D flux, and the reaction cross-section of H/D with O$_3$. The first two are obtained  from calibrations. The cross-section is obtained as follows. Let's suppose that at t=0 the ozone coverage on the surface is $\theta (0)=\theta_0$. After a certain H/D exposure time $t$ the ozone coverage becomes:
\begin{equation}
 \theta (t)=\theta_0 \exp(-\phi \sigma t)
\end{equation}
where $\phi$ is the beam flux of H/D in the unit of $cm^{-2}s^{-1}$, $\sigma$ is the cross-section for  H/D+O$_3$ reaction in units of $cm^{2}$, and $t$ is the exposure time. This indicates that the ozone yields should decay exponentially as H/D exposure times increases, which agrees very well with the experimental results shown in Figure \ref{fig:H_O3} and Figure \ref{fig:D_O3}. In Figure \ref{fig:log_ozone} the ozone yields are plotted in log$_e$ scale as a function of H/D exposure. The slope of the two linear fits are $-0.135\ min^{-1}$ and $-0.085\ min^{-1}$ for H and D, respectively. Using the flux calibration presented in Section \ref{sec:exp}, the cross-sections $\sigma$ for H and D are calculated to be: $\sigma_H=3.42\ \text{\AA}^2$ and $\sigma_D=1.52\ \text{\AA}^2$, respectively. These two cross-sections are smaller than the typical size of ozone molecules; therefore, we conclude that at 50 K on amorphous silicate surfaces H/D reacts with ozone via direct collision (the E-R mechanism) instead of the hot-atom mechanism. This conclusion can be generalized to other barrierless grain surface reactions that involve H/D because the non-hot-atom-hopping is independent of the other reactant.  This conclusion can also be generalized to higher surface temperatures, because at higher temperature the H/D atoms are even more likely to be reflected directly (indicates smaller cross-section) instead of staying on the surface temporarily and hopping around (indicates greater cross-section). To sum up: H/D takes part in chemical reactions on amorphous silicate surface via direct E-R mechanism when the surface temperature is higher than 50 K. 

\begin{figure}
\epsscale{1.1}
\plotone{./figures/log_ozone.eps}
\caption{Log plot of the ozone (amu 36) TPD yields as a function of H/D exposure time. The two straight lines show  linear fits. }
\label{fig:log_ozone}
\end{figure}

\subsection{Water formation and OH/OD desorption}
A qualitative analysis of OH/OD residence time can be obtained easily. In Reaction \ref{eq:o3+h}, the product OH/OD must stay on the surface until another H/D/H$_2$/D$_2$ arrives (Reactions (\ref{eq:oh+h}) and (\ref{eq:oh+h2})), otherwise no water can be formed; therefore, OH/OD should have a desorption peak temperature much higher than 50 K in order to remain on the surface. Thus, we see that the estimate of the desorption energy by  \citet{Allen1977} (1259 K) does not agree with our experiments because in that case OH or OD would desorb from the surface quickly at 50 K (residence time $\sim 0.1$ second). 

The attribution of the measured TPD spectra cannot be done unambiguously because of the existence of different isotopes and of fragmentation in the QMS. (Unfortunately, infrared spectroscopy is of no help here because of the low coverage and IR cross-sections). In Figure \ref{fig:H_O3} the amu 19 yields is always a fraction of the amu 20 yield, the ratio amu 19 /amu 20 is 0.24$\pm$0.03, which is what is obtained by measuring  the amu 17/amu 18 ratio of water in a separate experiment. We thus conclude that all the amu 19 signal  in Figure \ref{fig:H_O3} is due to H$_2$O fragmentation in the QMS; there is no OH direct desorption in the TPD stage. This indicates that the conversion from OH to water is very efficient.

The interpretation of the O$_3$+D experimental results in Figure \ref{fig:D_O3} is more complicated. The amu 20 which peaked at around 160 K may be due to three sources: OD formed from Reaction (\ref{eq:o3+h}),  fragmentation of HDO or D$_2$O, and D$_2 \, ^{16}$O coming from the beam line. The last one is already subtracted from the TPD yields in Figure \ref{fig:D_O3}. We now consider the first two possible scenarios:

(1) If at least part of the amu 20 signal is due to OD desorption, then the desorption temperature of OD is the same as water desorption, and the desorption energy is calculated to be around 410 meV (4760 K). By comparing Figure \ref{fig:H_O3} and Figure \ref{fig:D_O3} one can see that there is a strong isotopic effect between H$_2$O  and D$_2$O formations. In OH+H$_2$/H, the reaction proceeds very fast so that no OH desorption is seen, while in OD+D$_2$/D/H$_2$ (H$_2$ is from the background), unreacted OD desorb from the surface directly. OH+H and OD+D reactions are barrierless, and are not supposed to have a strong isotopic effect. The only possibility is that water forms efficiently via OH+H$_2$ by tunneling; but OD+D$_2$/H$_2$ forms at a much lower rate, and there is residual OD direct desorption. It's also likely that OD tends to desorb directly instead of diffusing and forming D$_2$O$_2$, because this molecule (amu 40) is not seen in the TPD spectra. 

(2) If none of the amu 20 in Figure \ref{fig:D_O3} is due to OD desorption, we estimate the desorption energy and the residence time of OH/OD in a different way.  We focus on  Figure \ref{fig:H_O3} and assume that OD requires the same desorption energy as OH. TPD spectra of different species cannot be compared directly, but a correction based on the speed of the desorbing species and on the cross-section for ionization in the detector is needed. Ozone (amu 54) desorption peaked at around 80 K while water (amu 20) desorption peaked at around 160 K. Taking into account the QMS detection efficiencies and considering that speed is correlated with temperature and mass as $v\sim \sqrt{T/m}$ (see Section \ref{sec:exp}), we have:
\[\frac{Y_{corr}(water)}{Y_{corr}(ozone)} =\frac{Y(H_2O) \sqrt{160/20}/4.00}{Y(O_3) \cdot \sqrt{80/54}/5.43} =3.15 \frac{Y(H_2O)}{Y(O_3)} \]
where $Y_{corr}$ represents corrected yields while $Y$ represents measured yields. Thus the water (amu 20) yield needs to be multiplied by 3.15 before comparing it with the ozone (amu 36) yield. According to the data in Figure \ref{fig:H_O3} and a simple calculation we see that the amount of water formed is 1.0$\pm$0.1 times the amount of ozone consumed, the error bar 0.1 is coming from an estimate in the uncertainty in the QMS measurements. Thus at least 90\% of the OH should be converted to water before desorption. In order to calculate the lower boundary of OH residence time at 50 K, we assume that all the H/H$_2$ can react with OH with 100\% probability as long as H/H$_2$ can strike an OH directly (E-R mechanism), and the reaction between H/H$_2$ and OH has the same cross-section as O$_3$+H. After OH is formed and remain on the surface for a time $t$, the fraction of OH that is not converted to water is $p(t)=\exp(-\phi(\text{H+H}_2) \sigma t)$, where $\phi(\text{H+H}_2)$ is the sum of H and H$_2$ beam flux. Letting $p(t)=0.1$ and using the values of the flux of H and H$_2$,  the residence time is calculated to  be $t=7$ minutes. This is the minimum required average residence time for OH on surface at 50 K in order to account for the water yields. The corresponding desorption energy is then at least 145 meV (1680 K). 

\citet{Dulieu2013} recently did a series of experiments to study the role of OH and OD in water formation on dust grain analogs. Experiments were done on a silicate sample kept in the range of 10 to 45 K. They deposited O$_2$ at 10 K and then irradiated the sample with  D atoms. They observed that some water (D$_2$O) desorbed during the irradiation with H or D, although the signal was just two times the noise level.  After irradiation they conducted a temperature programmed desorption (TPD) and observed the desorption of O$_2$, $D_2$O and D$_2$O$_2$, which is an intermediate product in the formation of water through the H+O$_2$ reaction. They did a similar experiment by depositing O$_3$ on the silicate surface and then by irradiating the sample with H or D at a temperature of 45K, which is higher than the temperature at which O$_2$ leaves the surface. They claimed that the H$_2$O or D$_2$O desorbed during irradiation and that there was no water left when they did a TPD experiment. This disagrees with our results. In our experiments we also record all the reaction products which may desorb upon formation. The O$_2$ signal increases above the background by about twice the noise level while the signal change for other species are not discernible. The difference between these two sets of experiments may be due to the different oxygen isotopes used in the experiments. In our experiments $^{18}$O is used in place of the much more naturally occurring $^{16}$O, so that the H$_2\,^{18}$O (amu 20) formed on the surface has a background contamination 10 times less than H$_2\,^{16}$O (amu 20) and the signal to noise ratio in our experiments is improved by at least 10 times compared with the experiments by Dulieu et al. where $^{16}$O was used. In addition, Dulieu et al. compared O$_3$+H at 45 K with O$_2$/O$_3$+H at 10 K, the latter may have a significant condensation of H$_2\,^{16}$O from the vacuum chamber onto the sample surface, which could be confused with the H$_2\,^{16}$O formed on the surface, thus making the comparison less convincing. 


\section{Astrophysical Implications}
\label{sec:sum}
There has been rapid progress in the laboratory to study the various routes to formation of water on a variety of surfaces of materials of interest in astrochemistry of the ISM. Most of the studies were concentrated on the feasibility of reaction paths on ices at low temperature. However, to predict how much water is produced in a given astrophysical environment one needs to know the water formation rate on surfaces and whether or not and how water is injected in the gas-phase. In this study, we explored the routes  of water formation on ``warm''   grains. On such grains, hydrogenation of O$_2$, one of the channels most considered today, is no longer viable because O$_2$ leaves the surface of grains at around 30-35K. At higher temperatures, the key to water formation is the radical OH. In this investigation, we looked at the formation of water via the reaction OH/OD+H/D/H$_2$/D$_2$ on a silicate surface at 50 K. OH/OD was obtained by O$_3$ hydrogenation/deuteration. We measured the cross-section for O$_3$+H/D to be $\sigma_H=3.42\ \text{\AA}^2$ and $\sigma_D=1.52\ \text{\AA}^2$, respectively, and determined that the mechanism of reaction is direct Eley-Rideal instead of hot-atom. 

 In the O$_3$+H/D experiments, differently from what is reported by \citet{Dulieu2013}, we have strong evidence that water doesn't desorb immediately upon formation. Our analysis suggests that it is probable that OH/OD desorbs in a TPD experiment at the same temperature as water does. In this case, the energy for desorption of OH/OD would be 410 meV (4760 K). Otherwise a lower boundary is suggested to be 145 meV (1680 K), which is more than 400 K higher than the previously accepted value of \citet{Allen1977}; this latter value is used in codes simulating the chemical evolution of interstellar clouds. Therefore, this new found value of OH binding energy could extends the effective water formation temperature on dust grains to much higher than 50 K. The question is then whether OH forms water by direct hydrogenation or by OH+OH$\rightarrow$H$_2$O$_2$ followed by H$_2$O$_2$ +H $\rightarrow$ H$_2$O+ OH. Thus, we need to compare the time it takes OH to diffuse and find another OH vs. the time it takes OH to react with an H atom impinging on the grain from the gas phase. 

Let's consider an OH on a surface. We assume \citep{He2014} that the energy barrier for diffusion to another site is $\alpha$ E$_{des}$. $\alpha$ ranges from 0.3 \citep{Bruch2007} to 0.7-0.8 \citep{Katz1999,Perets2007}; the first is value comes from a rule of thumb for weakly adsorbed atoms on well-ordered surfaces; the second value comes from data of H on amorphous silicates. Let's take a value of $\alpha$=0.5, as was recently proposed by \citet{Yildiz2013}. Assume T=50K, and $\nu$=10$^{12}$ sec$^{-1}$, then the rate of jumping off to another site is $f=\nu \;e^{-E_{diff}/k_BT}$ or $\sim 4\times10^{-9}$ sec$^{-1}$. Therefore to make a jump it takes 1/f=$2.5\times10^8$ sec. In a random walk, the OH will move a distance $ d \sim a \;\sqrt{N}$, where $a$ is the distance to the next site ($\sim$ 3 \AA) and $N$ is the number of jumps. The time it takes to move a distance $d$ is $5\times10^4$ $\sqrt{N}$ sec. The number of steps N required to find another OH depends on the coverage of OH. 

Let's now consider the reaction of OH on the dust grain with H coming from the gas phase. Assume that the number density of H n=10$^2$ atoms/cm$^3$. The number of H atoms hitting a grain of unit area per unit time is $\dot{N}=\frac{1}{4}nv$ and $v=\sqrt{\frac{3k_BT_H}m}$. For T$_H\sim$ 100K, we have v$\sim$10$^5$ cm/sec. Therefore, $\dot{N}$=$2.5\times10^5$ /sec/cm$^2$. 
\citet{Takahashi1999} calculated the diffusion length of a 70 K H atom on a 70 K ice surface to be 120 \AA before trapping. The H atom  visits 120 \AA/3 \AA=40 sites (assuming a distance of 3 \AA between sites) before trapping. The total number of sites visited by N atoms per second is $\sim 2.5\times10^{5}\times 40=10^7$ or 10$^{-8}$ of a layer, assuming a layer has 10$^{15}$ sites per cm$^2$. Therefore, 10$^{8}$ sec. would be required to visit every site by at least one H atom. In that time OH would have done about one successful jump. Even if we recognize that more time is required to visit all sites because of stochastic inefficiencies (multiple visits to sites, etc.), it is clear that  the OH+OH reaction to form water on warm grains is not competitive with the OH+H reaction, except in particular circumstances, such a high density of OH sites.

In the preceding paragraph we assumed that the atom diffuses before coming to rest.  We can also calculate the probability of an H atom to react with an OH is only via the Eley-Rideal mechanism. Assume a cross-section $\Sigma$ of $10 \;\text{\AA}^2$ for the H-OH reaction. Then, the rate of reaction is $\dot{N} \; \Sigma = 2.5\times10^5 \times 10^{-15}$ /sec = $2.5\times10^{-10}$ /sec or the time to get a successful reaction with cross-section $\Sigma$ is t $_{H-OH} \sim 4\times10^9$ sec. This reaction is competitive with the one of OH+OH interaction if the coverage of OH is such that the time for an OH to reach another OH is of the order of t$_{\text{OH-OH}} \sim 10^{10}$ sec.  $2\times10^5$ steps would be required; the corresponding minimum coverage of OH  is then $\sim d^2$ or $2\times10^{-10}\text{\AA}^2$ or $5\times10^{-4}$ of a monolayer. But under realistic conditions, H striking the bare surface would also travel some distance, thus making the minimum coverage for a OH+OH reaction much higher. Therefore, whether H reacts with OH via the Eley-Rideal or the hot-atom mechanisms, in the ISM conditions considered above, the OH+H reaction should be much more likely to proceed than the OH+OH reaction.

Experimentally, more work is needed to characterize the OH residence on grain analogs and to quantitatively evaluate the relative contributions of OH+H and OH+H$_2$ reactions to the formation of water on warm grains. Theoretically, formation rates of water on warm grains as deduced from experiments should be compared with photodissociation, photodesorption, and energetic charged particle interactions that would influence the overall H$_2$O production, desorption and destruction rates.

\acknowledgments
This work is supported by the NSF, Astronomy \& Astrophysics Division (Grants No. 0908108 and 1311958), NASA (Grant No. NNX12AF38G). We thank Dr. J.Brucato of the Astrophysical Observatory of Arcetri for providing the samples used in these experiments.
  
\bibliographystyle{apj}
\bibliography{ref}

\end{document}

